% This is samplepaper.tex, a sample chapter demonstrating the
% LLNCS macro package for Springer Computer Science proceedings;
% Version 2.21 of 2022/01/12
%
\documentclass[runningheads]{llncs}
%
\usepackage[T1]{fontenc}
% T1 fonts will be used to generate the final print and online PDFs,
% so please use T1 fonts in your manuscript whenever possible.
% Other font encondings may result in incorrect characters.
%
\usepackage{graphicx}
\usepackage{multirow}
\usepackage{ragged2e}
\usepackage{listings}
\usepackage{graphicx}
\graphicspath{ {./images/} }
%Define the listing package
\usepackage{listings} %code highlighter
\usepackage{color} %use color
\definecolor{mygreen}{rgb}{0,0.6,0}
\definecolor{mygray}{rgb}{0.5,0.5,0.5}
\definecolor{mymauve}{rgb}{0.58,0,0.82}
 
%Customize a bit the look
\lstset{ %
backgroundcolor=\color{white}, % choose the background color; you must add \usepackage{color} or \usepackage{xcolor}
basicstyle=\footnotesize, % the size of the fonts that are used for the code
breakatwhitespace=false, % sets if automatic breaks should only happen at whitespace
breaklines=true, % sets automatic line breaking
captionpos=b, % sets the caption-position to bottom
commentstyle=\color{mygreen}, % comment style
deletekeywords={...}, % if you want to delete keywords from the given language
escapeinside={\%*}{*)}, % if you want to add LaTeX within your code
extendedchars=true, % lets you use non-ASCII characters; for 8-bits encodings only, does not work with UTF-8
frame=single, % adds a frame around the code
keepspaces=true, % keeps spaces in text, useful for keeping indentation of code (possibly needs columns=flexible)
keywordstyle=\color{blue}, % keyword style
% language=Octave, % the language of the code
morekeywords={*,...}, % if you want to add more keywords to the set
numbers=left, % where to put the line-numbers; possible values are (none, left, right)
numbersep=5pt, % how far the line-numbers are from the code
numberstyle=\tiny\color{mygray}, % the style that is used for the line-numbers
rulecolor=\color{black}, % if not set, the frame-color may be changed on line-breaks within not-black text (e.g. comments (green here))
showspaces=false, % show spaces everywhere adding particular underscores; it overrides 'showstringspaces'
showstringspaces=false, % underline spaces within strings only
showtabs=false, % show tabs within strings adding particular underscores
stepnumber=1, % the step between two line-numbers. If it's 1, each line will be numbered
stringstyle=\color{mymauve}, % string literal style
tabsize=2, % sets default tabsize to 2 spaces
title=\lstname % show the filename of files included with \lstinputlisting; also try caption instead of title
}
%END of listing package%
 
\definecolor{darkgray}{rgb}{.4,.4,.4}
\definecolor{purple}{rgb}{0.65, 0.12, 0.82}
 
%define Javascript language
\lstdefinelanguage{JavaScript}{
keywords={typeof, new, true, false, catch, function, return, null, catch, switch, var, if, in, while, do, else, case, break},
keywordstyle=\color{blue}\bfseries,
ndkeywords={class, export, boolean, throw, implements, import, this},
ndkeywordstyle=\color{darkgray}\bfseries,
identifierstyle=\color{black},
sensitive=false,
comment=[l]{//},
morecomment=[s]{/*}{*/},
commentstyle=\color{purple}\ttfamily,
stringstyle=\color{red}\ttfamily,
morestring=[b]',
morestring=[b]"
}
 
\lstset{
language=JavaScript,
extendedchars=true,
basicstyle=\footnotesize\ttfamily,
showstringspaces=false,
showspaces=false,
numbers=left,
numberstyle=\footnotesize,
numbersep=9pt,
tabsize=2,
breaklines=true,
showtabs=false,
captionpos=b
}
% Used for displaying a sample figure. If possible, figure files should
% be included in EPS format.
%
% If you use the hyperref package, please uncomment the following two lines
% to display URLs in blue roman font according to Springer's eBook style:
%\usepackage{color}
%\renewcommand\UrlFont{\color{blue}\rmfamily}
%
\begin{document}
%
\title{Browser Fingerprinting Android Hybrid apps.}
%
%\titlerunning{Abbreviated paper title}
% If the paper title is too long for the running head, you can set
% an abbreviated paper title here
%
\author{Alimerdan Rahimov\inst{1} \and
Jyoti Prakash\inst{1} \and
Abhishek Tiwari\inst{1} \and
Christian Hammer\inst{1}}
%
\authorrunning{A. Rahimov et al.}
% First names are abbreviated in the running head.
% If there are more than two authors, 'et al.' is used.
%
\institute{University of Passau, Bayern, Germany \and
\email{rahimo01@ads.uni-passau.de, (jyoti.prakash, abhishek.tiwari, christian.hammer)@uni-passau.de}}

%
\maketitle              % typeset the header of the contribution
%
\begin{abstract}
Nowadays, most android applications embed web pages in WebViews, or the whole of the application can be built in WebViews. This method helps to create ones and run them on all platforms. Several frameworks already exist to help create hybrid apps. The security and privacy vulnerabilities of hybrid apps are much higher than those of native apps. They include security and privacy vulnerabilities inherited from the programming language in which they are written, as well as vulnerabilities that browsers pass on.  \\
 To our best knowledge, no general analysis of browser fingerprinting has previously been performed on hybrid applications. Initially, the analysis was done manually on popular hybrid apps such as Instagram, Facebook, Alibaba. We then automatically downloaded and tested 20,000 apps from AndroZoo database, from them  15,000 were hybrid apps. We mainly focused on privacy leaks of browser fingerprinting that could be imprinted. Most of our research focused on user-agent strings since we found them to be more interesting for browser fingerprints.



\keywords{Privacy, Browser Fingerprinting, Android}
\end{abstract}
%
%
%
\input{sections/intro}
\section{Motivation and Background}

Hybrid applications.
Hybrid applications are combination of native and web technologies. They allow web content run on their custom browsers. They use webview to run web content. By itself Webviews restricted by  a browser sandbox with very limited access to the native APIs of the mobile devices. But by creating JavaScriptInterfaces they can get acces to native apis. 
The difference of hybrid applications from native are crossplatform and cost effective. Populare frameworks are Cordova, Phonegap and Flutter. With the help of development tools browsers in hybrid applications can be customized. This can open new privacy issues in browser fingerprinting.  \\
Tracking \\
Cookies \\
Cookieles tracking\\
Advertising\\
Browser fingerprinting and its mitigation.\\
Passive fingerprinting \\
Active fingerprinting\\
Device fingerpinting\\
Device identifiers\\
Google identifier\\
Dynamic instrumentation\\
HTTP headers USer agent strings\\
\section{Methodology}
For our dynamic instrumentation we used frida instrumentation tool\cite{frida}. Frida is dynamic instrumentation toolkit which enables inject javascript snippets to applications of all platforms.


\includegraphics[width=\textwidth]{images/Proccess.PNG}
For instrument our android apps we have implemented the steps below:\\
1. Download frida server from release page.\\
2. Copy downloaded file into device or emulator.\\ 
3. Run the server inside the device.\\
After above steps frida server on our device is running. Then in the first step we wrote javascript code to inject to the android application and log webview loadUrl and OnloadResource functions. From this methods we log url, custom header and user-agent strings.\\
\begin{lstlisting}[language=JavaScript, caption=Instrumentation of Webview]    
    'use strict';

if (Java.available) {
    Java.perform(function() {
        //android package name 
        const ActivityThread = Java.use('android.app.ActivityThread');
        var context = ActivityThread.currentApplication().getApplicationContext();
        var packagename = context.getPackageName();
        
        var WebView = Java.use("android.webkit.WebView");
        //Instrumenting webview.loadUrl 
        WebView.loadUrl.overload('java.lang.String').implementation = function(url) {
            this.loadUrl(url);
            //Sending to python code record logs
            send({
                packageName: packagename,
                method: "loadUrl",
                Url: url,
                Header: "",
                userAgent: this.getSettings().getUserAgentString()
            });
        }
           //Instrumenting webview.loadUrl with custom headers
        WebView.loadUrl.overload('java.lang.String', 'java.util.Map').implementation = function(url, header) {
            this.loadUrl(url, header);
               //Unmapping custom headers
            var keyset = p1.keySet();
            var it = keyset.iterator();
            while (it.hasNext()) {
                var keystr = it.next().toString();
                var valuestr = p1.get(keystr).toString();
            }
         //Sending to python code record logs
            send({
                method: "loadUrlHeader",
                Url: url,
                Header: header,
                userAgent: this.getSettings().getUserAgentString()
            });
        }
        var WebViewClient = Java.use("android.webkit.WebViewClient");
          //Instrumenting webviewclient.onLoadResource
        WebViewClient.onLoadResource.overload('android.webkit.WebView', 'java.lang.String').implementation = function(webview, url) {
            send({
                packageName: packagename,
                method: "onLoadResource",
                Header: "",
                Url: url,
                userAgent: webview.getSettings().getUserAgentString()
            });
            this.onLoadResource(webview, url);
        }
    });

}
      \end{lstlisting}    

Second step is for automatization of injection proccess in the python scripts.In Listing 2 you can see our code:\\
\begin{lstlisting}[language=Python, caption=Installation] 
import time

from ppadb.client import Client as AdbClient
import re
import os
import csv
import frida
path = 'path/to/the/applications/folder'
path1='path/to/the/tested/applications/folder'

#Connects to device
def connect():
  
    client = AdbClient(host="127.0.0.1", port=5037) 
    devices = client.devices()
    if len(devices) == 0:
        print('No devices')
        quit()
    device = devices[0]
    print(f'Connected to {device}')
    return device, client

def search_package_in_avd(device):
    command = device.shell('pm list packages -3 '+'|cut -f 2 -d '+':')
    packages=re.split(':|\r|\n',command)
    for package in packages:
     print(package+"\n")
    if not packages:
        return ""
    else:
        return packages
#Reads apk file names in folder        
def read_files():
    files = os.listdir(path)
    return files

def install_package(package):
    try:
     device.install(path+'/'+package)
     print(package+" installed ")
     return True
    except Exception as e:
     print("Error"+str(e))
     os.remove(path+"/"+package)
     return False
    
def uninstall_package(device):
     packages=search_package_in_avd(device)
     for package in packages:
        device.uninstall(package)
        print(package+" uninstalled")
#Initialize csv file writing
def file_init():
    header = ["packageName","package","method","header","url","user agent"]
    file = open('user_agents.csv', 'a')
    writer = csv.writer(file)
    writer.writerow(header)    
    return file,writer
#Write into csv file
def file_open1():
    file = open('user_agents.csv', 'a')
    writer = csv.writer(file)
    return file,writer    
def add_rows(writer,data):
    writer.writerow(data)
#frida instrumentation
def frida_instument():
  try:
    device_frida = frida.get_usb_device()
    f_package=search_package_in_avd(device)[0];
    pid = device_frida.spawn([f_package])
    session = device_frida.attach(pid)
    script = session.create_script(open("instrumentJs.js").read())
    script.on("message", on_message)
    script.load()
    device_frida.resume(pid)
    time.sleep(5)
  except Exception as e:
    print("ERROR"+" "+str(e))
#Listens messages from injected js code and records to file  
def on_message(message, data):
    print("frida")
    if 'payload' in message:
        payload = message['payload']  
        if 'Url' in payload:
            print("inFrida")
            data=[payload['packageName'],package,payload['method'],payload['Header'],payload['Url'], payload['userAgent']]
            file,writer=file_open1() 
            add_rows(writer,data)
            file.close()
            
device=None
writer=None
file=None
package=None
f_package=None
if __name__ == '__main__':
    file,writer=file_init() 
    device, client = connect()     
    uninstall_package(device)
    apks=read_files() 
    for apk in apks:
       x=install_package(apk)
       package=apk
       if(x):
        frida_instument()
        uninstall_package(device)
        os.replace(path+"/"+apk,path1+"/"+apk)       
    file.close()
    \end{lstlisting}   




Above code implements below functions:\\
1. Application installation and removal \\
2. Instruments the webview to Log information .\\

Third step is processing obtained data:
\begin{lstlisting}[language=Python, caption=Proccessing the obtained data] 
import os
import csv
url={}
useragent={}

with open('user_agents.csv', newline='') as csvfile:
        reader = csv.DictReader(csvfile)
        for row in reader:
         #Grouped unencrypted traffic by packages       
         if row["packageName"] not in url:
          if "http://" in row["url"]:
           url[row["packageName"]]=[]
           url[row["packageName"]].append(row["url"])
         elif "http://" in row["url"] :
          url[row["packageName"]].append(row["url"])
         #Grouped user agents by  packages  
         if  row["user agent"] not in useragent:
           useragent[row["user agent"]]=[]
           useragent[row["user agent"]].append(row["packageName"])
         else:
           useragent[row["user agent"]].append(row["packageName"])       
      
  
       
       
        file = open('url.csv', 'a')
        writer = csv.writer(file)  
        header = ["packageName","url"] 
        writer.writerow(header)        
        for key, value in url.items():
            data=[key,value]
            writer.writerow(data)
                
                
        file = open('useragents1.csv', 'a')
        writer = csv.writer(file)  
        header = ["useragents","package"]
        writer.writerow(header)   
        for key, value in useragent.items():
            data=[key,"|",value]
            writer.writerow(data)       

   \end{lstlisting}  
       

 The data collection was done in an emulator. In this emulator, I signed up with a dummy email address on google. This is how I know if an app is using my google ads id, my GSF id, or some personal settings.  
 We used the adb package to install apps on the emulator. 



After collecting all the data we created a simple python program to analyze the data. 
1. Grouping user agent rows.
User agent grouping.  To review the different user agents. 
2. Find unencrypted traffic. 
Find unencrypted traffic and match it with a packet. We have separated the unencrypted packet that sends sensitive information and that is not malicious.

\usepackage{multirow}
\usepackage{ragged2e}
\section{Evaluation}
In this section we describe the results of our evaluation.

\subsection{Data collection.}
In our dataset we have 5 most popular hybrid apps from google play store and 20000 applications from AndroZoo dataset\cite{allix2016androzoo}. To download hybrid applications from AndrooZoo we reffered to hash of applications from dataset of Babelview  research\cite{rizzo2018babelview}.\\
Data collection was conducted on Acer Aspire E1-572G with a proccessor Intel Core i7-4500U . We evaluated 20000 applications from AndroZoo.The evaluated data set contains user-agents, headers and urls.

\par
\subsubsection{Case Study 1.Chromium Webview user-agent leakage.}\par

 Using the webview inside the applications makes usage of website links easier as the user will not leave the application. So nowadays most of applications uses  Webview. However using this makes it impact on users privacy. Below we can look at default chromium webview user-agent strings:\\
 
         "Mozilla/5.0 (Linux; Android 9; SM-A505FN Build/PPR1.180610.011; wv)                     AppleWebKit/537.36 (KHTML, like Gecko) Version/4.0  Chrome/99.0.4844.88              Mobile Safari/537.36 "         \\
         \par

User-agent leaks users phone model and build number of the device. This bug was mentioned in article in 2015\cite{nightwatch} and in chromium forum\cite{forum}. In 2015 chrome browser was also leaking phonel model and build number. In 2018 chrome browser removed the build number from user agent strings'\cite{nightwatch1}. But still, the phone model was in the user agent strings. Now chrome browser user-agent strings looks like this:\\

                "{Mozilla/5.0 (Linux; Android 9; SM-A505FN) AppleWebKit/537.36 (KHTML,                like Gecko) Chrome/98.0.4758.87 Mobile Safari/537.36}" \\
\par
 Since Chrome from version 93 google chrome's privacy sandbox project started providing reduced user agent strings to limit the leakage of sensitive information and protect from passive fingerprinting. We can manually enable it in Chrome by: "{chrome://flags/#reduce-user-agent}"
After enabling this plugin user-agents will be default in all android devices:\\

       "{Mozilla/5.0 (Linux; Android 10; K) AppleWebKit/537.36 (KHTML, like  Gecko) Chrome/93.0.0.0 Mobile Safari/537.36}" \\
       
The correct browser and platform information website can get with the User agent hints\cite{useragentred}.
Our motivation to start this analysis is to check what kind of information is send with user agent field in hybrid browsers. As it can be set by a developer and contain sensitive information which can uniquely identify a user. 

Case study 1. How am I unique?\par
Hybrid applications allow developers to create browsers within applications and customize them to their needs. One such customization is the user agent strings. Some applications reveal detailed information about the device. Webviews leak sensitive information by default, as described in the previous subcsection . However, some applications add detailed information about the device, making it vulnerable to passive fingerprinting. One of the most popular apps is Instagram. It has a built-in browser. The built-in browser is invoked when you click on a link inside the Instagram app. The user agent of the built-in Instagram browser is shown below:\par

           {"Mozilla/5.0 (Linux; Android 9; SM-A505FN Build/PPR1.180610.011; wv) AppleWebKit/537.36 (KHTML, like Gecko) Version/4.0  Chrome/99.0.4844.88 Mobile Safari/537.36 Instagram 229.0.0.17.118  Android (28/9; 420dpi; 1080x2131;           samsung; SM-A505FN; a50; exynos9610;en\_DE; 360889116)"}
\\
The user-agent string of Instagram does not conform to the RFC standard. I quote from RFC7231 in section 5.5.3\cite{rfc}:

    {" A user agent SHOULD NOT generate a User-Agent field  containing needlessly fine-grained detail and SHOULD limit the addition of  subproducts by third parties.  Overly long and detailed User-Agent  field values increase request latency and the risk of a user being identified against their wishes ("fingerprinting"). "}

Below we have detailed the information presented on the Instagram's user-agent:\\
1. Phone model leakage: (samsung; SM-A505FN; a50;).\\
Webviews leaks the phone model by default as we mentioned in previous subsection. However, in addition, Instagram adds its own information about the device. Even if the Webviews privacy bugs in Chromium are fixed, Instagram will still leak phone model information.\\
2.  Phone Build number leakage: (Build/PPR1.180610.011;)\\
It is the default Webview security vulnerability we covered in the previous subsection.\\
3. Android version :(Android 9;28/9;) \\
The Android version is already in the default user agent string in Webview. However, Instagram added the SDK version and the android version.\\
4. DPI and resolution (420dpi; 1080x2131;)\\
This information can be used to determine screen size. But it is unlikely that browsers will need this information nowadays.
5. Instagram version and variant (Instagram 229.0.0.17.118;360889116)\\
6. Processor name (exynos9610;)\\
This is interesting information to have the processor name in the user agent.\\
7. Localization information. \\
The localization information in the user agent is redundant. Because this information can be obtained by the server using HTTP\_ACCEPT\_Header.
We checked how unique our user agent string is. We used the dataset from A Project of the Electronic Frontier Foundation\cite{tracks}. Table 1 shows the uniqueness of the user agent strings in the dataset. You can compare the uniqueness of Instagram user agent strings with standard mobile browsers.
When we compared the dataset size was x=214144.

\begin{tabular}{ |p{3cm}||p{3cm}|p{3cm}|p{3cm} |}
 \hline
 \multicolumn{4}{|c|}{User-agent uniqeness} \\
 \hline
Application & User-agent& Bits of identifying information& One in x browsers have this value\\
 \hline
Mozilla & Mozilla/5.0 (Android 9; Mobile; rv:98.0) Gecko/98.0 Firefox/98.0& 9.88 & 942.93\\
 \hline
Chrome & Mozilla/5.0 (Linux; Android 10; K) AppleWebKit/537.36 (KHTML, like Gecko) Chrome/99.0.0.0 Mobile Safari/537.36 & 7.97 & 250.63\\
 \hline
Instagram & Mozilla/5.0 (Linux; Android 9; SM-A505FN Build/PPR1.180610.011; wv) AppleWebKit/537.36 (KHTML, like Gecko) Version/4.0  Chrome/99.0.4844.88 Mobile Safari/537.36 Instagram 229.0.0.17.118  Android (28/9; 420dpi; 1080x2131;           samsung; SM-A505FN; a50; exynos9610;en\_DE; 360889116) &17.91 & 214144.0\\
\hline
\end{tabular}\\
  Table 1 
 \\
From the table we can see using fine-grained information increases user uniqueness, which leads to the privacy leakage.

Case study 2. Phone model 
By default all webviews leaks phone model. Security advisors several times pointed out that developers need's to override user agent strings to hide phone model and build number\cite{forum}\cite{nightwatch}\cite{nightwatch1}. But from our research we have found none of applications hide them. We have tested about 20000 applications from androzoo and 15000[~] are hybrid apps. 
One example is Alibaba website. Alibaba has hybrid application.The privacy vulnerability of website is after login to the account  , account number attached to the cookies. As it stays same always,if user uses Alibaba hybrid applications or other in-app browsers to Login the site, server will now about user changed phone.It has several privacy issus:
1. How frequently you change your phone?
2. How is your financial status?(As in some countries mobile phones shows your financial status).
But if user will use privacy oriented browsers example Mozilla,Tor or Chrome with enabled user-agent reduction this privacy issues will not happen. 

Case study 3. Build number 
Build number on mobile is security Vulnerability. There are several researches has done that with the help of build number the phone can be exploited\cite{nightwatch02}.

case study 4. Unique Id in user agent.
In our dataset some hybrid applications uses unique id in their user agents. Which is not  RFC7230 standart:\\

"A sender SHOULD limit generated product identifiers to what is
   necessary to identify the product; a sender MUST NOT generate
   advertising or other nonessential information within the product
   identifier.  A sender SHOULD NOT generate information in
   product-version that is not a version identifier (i.e., successive versions of the same product name ought to differ only in the product-version portion of the product identifier)." \\
   
   
An example of this net.giosis.shopping.id package:
\\
\Centering
 "Mozilla/5.0 (Linux; Android 11; Android SDK built for x86 Build/RSR1.210210.001.A1; wv) AppleWebKit/537.36 (KHTML, like Gecko) Version/4.0 Chrome/83.0.4103.106 Mobile Safari/537.36 AndroidGmarket Qoo10 ID\_3.6.2\_133(GMKTV2\_ZlRnG1XAIzgwoC3OBe0hNjV4Pfmya
 C5RAIBqY+mkcipUGsSIiB19AyfIHQY1msEafG/xGz9RIS4=;
 Android SDK built for x86;11;en\_US;2000010476)"
Unique id is added for tracking purposess. We have tested that unique id is connected device itself so after reinstalling this UID stays same. 





\input{sections/relatedwork}
\input{sections/conclusion}
%
% ---- Bibliography ----
%
% BibTeX users should specify bibliography style 'splncs04'.
% References will then be sorted and formatted in the correct style.
%
 \bibliographystyle{splncs04}
 \bibliography{RelatedWork}
%
\end{document}
