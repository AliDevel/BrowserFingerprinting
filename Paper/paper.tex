% This is samplepaper.tex, a sample chapter demonstrating the
% LLNCS macro package for Springer Computer Science proceedings;
% Version 2.21 of 2022/01/12
%
\documentclass[runningheads]{llncs}
%
\usepackage[T1]{fontenc}
% T1 fonts will be used to generate the final print and online PDFs,
% so please use T1 fonts in your manuscript whenever possible.
% Other font encondings may result in incorrect characters.
%
\usepackage{graphicx}
\usepackage{multirow}
\usepackage{ragged2e}
\usepackage{listings}
\usepackage{graphicx}
\graphicspath{ {./images/} }
%Define the listing package
\usepackage{listings} %code highlighter
\usepackage{color} %use color
\definecolor{mygreen}{rgb}{0,0.6,0}
\definecolor{mygray}{rgb}{0.5,0.5,0.5}
\definecolor{mymauve}{rgb}{0.58,0,0.82}
 
%Customize a bit the look
\lstset{ %
backgroundcolor=\color{white}, % choose the background color; you must add \usepackage{color} or \usepackage{xcolor}
basicstyle=\footnotesize, % the size of the fonts that are used for the code
breakatwhitespace=false, % sets if automatic breaks should only happen at whitespace
breaklines=true, % sets automatic line breaking
captionpos=b, % sets the caption-position to bottom
commentstyle=\color{mygreen}, % comment style
deletekeywords={...}, % if you want to delete keywords from the given language
escapeinside={\%*}{*)}, % if you want to add LaTeX within your code
extendedchars=true, % lets you use non-ASCII characters; for 8-bits encodings only, does not work with UTF-8
frame=single, % adds a frame around the code
keepspaces=true, % keeps spaces in text, useful for keeping indentation of code (possibly needs columns=flexible)
keywordstyle=\color{blue}, % keyword style
% language=Octave, % the language of the code
morekeywords={*,...}, % if you want to add more keywords to the set
numbers=left, % where to put the line-numbers; possible values are (none, left, right)
numbersep=5pt, % how far the line-numbers are from the code
numberstyle=\tiny\color{mygray}, % the style that is used for the line-numbers
rulecolor=\color{black}, % if not set, the frame-color may be changed on line-breaks within not-black text (e.g. comments (green here))
showspaces=false, % show spaces everywhere adding particular underscores; it overrides 'showstringspaces'
showstringspaces=false, % underline spaces within strings only
showtabs=false, % show tabs within strings adding particular underscores
stepnumber=1, % the step between two line-numbers. If it's 1, each line will be numbered
stringstyle=\color{mymauve}, % string literal style
tabsize=2, % sets default tabsize to 2 spaces
title=\lstname % show the filename of files included with \lstinputlisting; also try caption instead of title
}
%END of listing package%
 
\definecolor{darkgray}{rgb}{.4,.4,.4}
\definecolor{purple}{rgb}{0.65, 0.12, 0.82}
 
%define Javascript language
\lstdefinelanguage{JavaScript}{
keywords={typeof, new, true, false, catch, function, return, null, catch, switch, var, if, in, while, do, else, case, break},
keywordstyle=\color{blue}\bfseries,
ndkeywords={class, export, boolean, throw, implements, import, this},
ndkeywordstyle=\color{darkgray}\bfseries,
identifierstyle=\color{black},
sensitive=false,
comment=[l]{//},
morecomment=[s]{/*}{*/},
commentstyle=\color{purple}\ttfamily,
stringstyle=\color{red}\ttfamily,
morestring=[b]',
morestring=[b]"
}
 
\lstset{
language=JavaScript,
extendedchars=true,
basicstyle=\footnotesize\ttfamily,
showstringspaces=false,
showspaces=false,
numbers=left,
numberstyle=\footnotesize,
numbersep=9pt,
tabsize=2,
breaklines=true,
showtabs=false,
captionpos=b
}
% Used for displaying a sample figure. If possible, figure files should
% be included in EPS format.
%
% If you use the hyperref package, please uncomment the following two lines
% to display URLs in blue roman font according to Springer's eBook style:
%\usepackage{color}
%\renewcommand\UrlFont{\color{blue}\rmfamily}
%
\begin{document}
%
\title{Browser Fingerprinting Android Hybrid apps.}
%
%\titlerunning{Abbreviated paper title}
% If the paper title is too long for the running head, you can set
% an abbreviated paper title here
%
\author{Alimerdan Rahimov\inst{1} \and
Jyoti Prakash\inst{1} \and
Abhishek Tiwari\inst{1} \and
Christian Hammer\inst{1}}
%
\authorrunning{A. Rahimov et al.}
% First names are abbreviated in the running head.
% If there are more than two authors, 'et al.' is used.
%
\institute{University of Passau, Bayern, Germany \and
\email{rahimo01@ads.uni-passau.de, (jyoti.prakash, abhishek.tiwari, christian.hammer)@uni-passau.de}}

%
\maketitle              % typeset the header of the contribution
%
\begin{abstract}
Nowadays, most android applications embed web pages in WebViews, or the whole of the application can be built in WebViews. This method helps to create ones and run them on all platforms. Several frameworks already exist to help create hybrid apps. The security and privacy vulnerabilities of hybrid apps are much higher than those of native apps. They include security and privacy vulnerabilities inherited from the programming language in which they are written, as well as vulnerabilities that browsers pass on.  \\
 To our best knowledge, no general analysis of browser fingerprinting has previously been performed on hybrid applications. Initially, the analysis was done manually on popular hybrid apps such as Instagram, Facebook, Alibaba. We then automatically downloaded and tested 20,000 apps from AndroZoo database, from them  15,000 were hybrid apps. We mainly focused on privacy leaks of browser fingerprinting that could be imprinted. Most of our research focused on user-agent strings since we found them to be more interesting for browser fingerprints.



\keywords{Privacy, Browser Fingerprinting, Android}
\end{abstract}
%
%
%
\input{sections/intro}
\section{Motivation and Background}


\section{Methodology}


\subsection{Data collection.}
We chose A Project of the Electronic Frontier Foundation as our experimental setup. Their project aims to investigate browsers for fingerprinting. They use the fingerprintJS library to perform various fingerprinting tests. During our testing, they had a dataset consisting of 217,923 browser fingerprints. They obtain and combine different browser configurations for fingerprinting. These include. 
    User agent, HTTP\_ACCEPT headers, browser plugin details, time zone offset, time zone, screen size and color depth, system fonts, are cookies enabled? limited supercookie test, canvas fingerprint hash, WebGL fingerprint hash, WebGL Vendor \texttt{\&amp;} Renderer, DNT Header Enabled?
Language, platform, touch support, AdBlocker used, AudioContext fingerprint, processor class, hardware parallelism, device memory (GB).
This dataset we used to get to know the difference in fingerprinting between mobile browsers and hybrid applications. We compared mobile browsers chrome, firefox, webview, and Instagram in app with these data sets. Oliver has done previous research cite comparing mobile and desktop browsers. In this study, however, we focus on the leakage of private information by fingerprinting. 
Some Hybrid applications user agents we cannot check with the eff.org project so we used our own intrumentation to check their user agents.
In our manually curated dataset we have 5 most popular hybrid apps from google play store and 20000 applications from AndroZoo dataset\cite{allix2016androzoo}. To download hybrid applications from AndrooZoo we reffered to hash of applications from dataset of Babelview  research\cite{rizzo2018babelview}.\\
Data collection was conducted on Acer Aspire E1-572G with a proccessor Intel Core i7-4500U . We evaluated 20000 applications from AndroZoo.The evaluated data set contains user-agents, headers and urls.

\section{Evaluation}
In this section we describe the results of our evaluation.

We compared fingerprintablity of mobile browsers chrome, firefox, webview, and Instagram in app browser. This\cite{oliver2018fingerprinting}  research  compares mobile and desktop browsers respect to fingerprinting. Our research focuses on the leakage of sensitive information which can be fingerprinted. 

 \textbf{RSQ1. What is the difference between hybrid apps and mobile browsers with respect to leaking sensitive information while browser fingerprinting?} \par
During our comparison of mobile browsers and webview we noticed leakage of sensitive information. \par
1. The first leakage is in the user agent strings.  
\subsubsection{Case Study 1.Privacy leakage unique to WebViews (what information WebViews leak uniquely?) + Build number of mobile + Unique ID in user agent}
\par

 Using the webview inside the applications makes usage of website links easier as the user will not leave the application. So nowadays most of applications uses  Webview. However using this makes it impact on users privacy. Below we can look at default chromium webview user-agent strings:\\
 
         "Mozilla/5.0 (Linux; Android 9; SM-A505FN Build/PPR1.180610.011; wv)                     AppleWebKit/537.36 (KHTML, like Gecko) Version/4.0  Chrome/99.0.4844.88              Mobile Safari/537.36 "         \\
         \par

User-agent leaks users phone model and build number of the device. This bug was mentioned in article in 2015\cite{nightwatch} and in chromium forum\cite{forum}. In 2015 chrome browser was also leaking phonel model and build number. In 2018 chrome browser removed the build number from user agent strings'\cite{nightwatch1}. But still, the phone model was in the user agent strings. Now chrome browser user-agent strings looks like this:\\

                "{Mozilla/5.0 (Linux; Android 9; SM-A505FN) AppleWebKit/537.36 (KHTML,                like Gecko) Chrome/98.0.4758.87 Mobile Safari/537.36}" \\
\par
 Since Chrome from version 93 google chrome's privacy sandbox project started providing reduced user agent strings to limit the leakage of sensitive information and protect from passive fingerprinting. We can manually enable it in Chrome by: "{chrome://flags/\#reduce-user-agent}"
After enabling this plugin user-agents will be default in all android devices:\\

       "{Mozilla/5.0 (Linux; Android 10; K) AppleWebKit/537.36 (KHTML, like  Gecko) Chrome/93.0.0.0 Mobile Safari/537.36}" \\
       
The correct browser and platform information website can get with the User agent hints\cite{useragentred}.
Our motivation to start this analysis is to check what kind of information is send with user agent field in hybrid browsers. As it can be set by a developer and contain sensitive information which can uniquely identify a user. 
Hybrid applications allow developers to create browsers within applications and customize them to their needs. One such customization is the user agent strings. Some applications reveal detailed information about the device. Webviews leak sensitive information by default, as described in the previous subcsection . However, some applications add detailed information about the device, making it vulnerable to passive fingerprinting. One of the most popular apps is Instagram. It has a built-in browser. The built-in browser is invoked when you click on a link inside the Instagram app. The user agent of the built-in Instagram browser is shown below:\par

           {"Mozilla/5.0 (Linux; Android 9; SM-A505FN Build/PPR1.180610.011; wv) AppleWebKit/537.36 (KHTML, like Gecko) Version/4.0  Chrome/99.0.4844.88 Mobile Safari/537.36 Instagram 229.0.0.17.118  Android (28/9; 420dpi; 1080x2131;           samsung; SM-A505FN; a50; exynos9610;en\_DE; 360889116)"}
\\
The user-agent string of Instagram does not conform to the RFC standard. I quote from RFC7231 in section 5.5.3\cite{rfc}:

    {" A user agent SHOULD NOT generate a User-Agent field  containing needlessly fine-grained detail and SHOULD limit the addition of  subproducts by third parties.  Overly long and detailed User-Agent  field values increase request latency and the risk of a user being identified against their wishes ("fingerprinting"). "}

Below we have detailed the information presented on the Instagram's user-agent:\\
1. Phone model: (samsung; SM-A505FN; a50;).\\
Webviews leaks the phone model by default as we mentioned in previous subsection. However, in addition, Instagram adds its own information about the device. Even if the Webviews privacy bugs in Chromium are fixed, Instagram will still leak phone model information.\\
2.  Phone Build number: (Build/PPR1.180610.011;)\\
It is the default Webview security vulnerability we covered in the previous subsection.\\
3. Android version :(Android 9;28/9;) \\
The Android version is already in the default user agent string in Webview. However, Instagram added the SDK version and the android version.\\
4. DPI and resolution (420dpi; 1080x2131;)\\
This information can be used to determine screen size. But it is unlikely that browsers will need this information nowadays.
5. Instagram version and variant (Instagram 229.0.0.17.118;360889116)\\
6. Processor name (exynos9610;)\\
This is interesting information to have the processor name in the user agent.\\
7. Localization information. \\
The localization information in the user agent is redundant. Because this information can be obtained by the server using HTTP\_ACCEPT\_Header.
We checked how unique our user agent string is. We used the dataset from A Project of the Electronic Frontier Foundation\cite{tracks}. Table 1 shows the uniqueness of the user agent strings in the dataset. You can compare the uniqueness of Instagram user agent strings with standard mobile browsers.
When we compared the dataset size was x=214144.

\begin{tabular}{ |p{3cm}||p{3cm}|p{3cm}|p{3cm} |}
 \hline
 \multicolumn{4}{|c|}{User-agent uniqeness} \\
 \hline
Application & User-agent& Bits of identifying information& One in x browsers have this value\\
 \hline
Mozilla & Mozilla/5.0 (Android 9; Mobile; rv:98.0) Gecko/98.0 Firefox/98.0& 9.88 & 942.93\\
 \hline
Chrome & Mozilla/5.0 (Linux; Android 10; K) AppleWebKit/537.36 (KHTML, like Gecko) Chrome/99.0.0.0 Mobile Safari/537.36 & 7.97 & 250.63\\
 \hline
Instagram & Mozilla/5.0 (Linux; Android 9; SM-A505FN Build/PPR1.180610.011; wv) AppleWebKit/537.36 (KHTML, like Gecko) Version/4.0  Chrome/99.0.4844.88 Mobile Safari/537.36 Instagram 229.0.0.17.118  Android (28/9; 420dpi; 1080x2131;           samsung; SM-A505FN; a50; exynos9610;en\_DE; 360889116) &17.91 & 214144.0\\
\hline
\end{tabular}\\
  Table 1 
 \\
From the table we can see using fine-grained information increases user uniqueness, which leads to the privacy leakage.
By default all webviews leaks phone model. Security advisors several times pointed out that developers need's to override user agent strings to hide phone model and build number\cite{forum}\cite{nightwatch}\cite{nightwatch1}. But from our research we have found none of applications hide them. We have tested about 20000 applications from androzoo and 7345[~] are hybrid apps. 
One example is the Alibaba website. Alibaba has a hybrid app. A website privacy vulnerability is that the account number is attached to cookies after logging into an account.(But most of applications adds account id to the cookie I checked with instagram and also with several unpopular apps) As it stays the same always, if the user uses Alibaba hybrid applications or other in-app browsers to log in to the site, the server will know that the user changed phone. It has privacy issues:\\
1. How frequently do you change your phone?
2. What is your financial situation? (As in some countries, mobile phones show your financial situation).
But if user will use privacy oriented browsers example Brave,Tor,Mozilla or Chrome with enabled user-agent reduction this privacy issue will not happen. 
Build number on mobile is security Vulnerability. There are several researches has done that with the help of build number the phone can be exploited\cite{nightwatch02}.If attacker knows the build number of phone he can exploit the phone. 

2. HTTP\_ACCEPT Language.
First of all lets start with what is ACCEPT-language. 
Shortly describing accept language header is expected to be added in all http requests. It indicates the natural language and locale that the client prefers. Server accepts this header then uses content negotiation to select one of the proposals and informs to the client with the content-language response header. 
 If there is no content matching language server sends back error code 406 (Not acceptable). But in most such cases Accept language header is ignored. 
Examples of ACCEP-Language:
Accept-Language: de
Accept-Language: fr-CH, fr;q=0.9, en;q=0.8, de;q=0.7, *;q=0.5
Accept-Language: *


The first example requests just for German content. In the second example more specific   tags  are used with priority list. It requests Swiss regional variant of French language, if it doesn't have then french else  english then german. 

From our best of knowledge we checked chrome, mozilla and hybrid apps whether what do they include in accept language header. 
We find all installed languages on phone appeared on Accept language header. 
Example:

But according to the page Accept-Language I am quoting it: 
 'Browsers set required values for this header according to their active user interface language. Users rarely change it, and such changes are not recommended because they may lead to fingerprinting.'

This means if users installs more languages to his phone, he can be uniquely identified in some cases. 
However it can be mitigated by going to settings and configure browser languages. 
But it is not the case with hybrid applications which uses webview.  Users don't have control on it. 
In some cases there is several privacy concerns:
1. Vulnerability of being uniquely identified.
2. Vulnerability of being identified your origin, ethnicity or nation.
For above cases we can use these example:
Sending en  ru  de tk
combination of This lanugages gives uniquesess of the user because not much people knows Turkmen language.  Uniqueness from coveryourtracks:

The second vulnerabilty which can arise is identification of users origin, ethnicity or nation. In this case we can see the combination of russian and turkmen languages. As it may show the user origin from Turkmenistan.(Do we need to add this?) 

 
 




case study 2.  Uniquely identify app users — by using phone number/unique user account 

In our dataset ]number of apps] hybrid applications uses unique id in their user agents. Which is not  RFC7230 standart:\\

"A sender SHOULD limit generated product identifiers to what is
   necessary to identify the product; a sender MUST NOT generate
   advertising or other nonessential information within the product
   identifier.  A sender SHOULD NOT generate information in
   product-version that is not a version identifier (i.e., successive versions of the same product name ought to differ only in the product-version portion of the product identifier)." \\
   
   
An example of this net.giosis.shopping.id package:
\\

               "Mozilla/5.0 (Linux; Android 11; Android SDK built for x86                        Build/RSR1.210210.001.A1; wv) AppleWebKit/537.36 (KHTML, like Gecko) Version/4.0 Chrome/83.0.4103.106 Mobile Safari/537.36 AndroidGmarket Qoo10 ID\_3.6.2\_133(\\GMKTV2\_ZlRnG1XAIzgwoC3OBe0hNjV4Pfmya
 C5RAIBqY+mkcipUGsSIiB\\19AyfIHQY1msEafG/xGz9RIS4=;
 Android SDK built for x86;11;en\_US;2000010476)"
 \\
 
A unique identifier is added for identify device uniquely. We have verified that the unique identifier is the connected device itself, so this UID remains unchanged after reinstallation.
This type of unique identifier can be attached as a user request header, which is an example in the Alibaba hybrid app. Users are unaware of this tracking mechanism and have no control over it. 

Case study. Phone model leakage in social hybrid apps.
For our example, we chose the Instagram application. In this example, we created a simple web page storing the Instagram accountId, accountSpecificLink, and user-agent of the user.
 We created a specific url for each Instagram account and sent them to a user. After the user clicked on the account, we could tell which user was using which phone. \\
 \begin{tabular}{ |p{3cm}|p{5cm}|p{5cm}|p{5cm} |}
 \hline
 \multicolumn{4}{|c|}{Accounts and phones} \\
 \hline
Account & AccountSpecificlink & User-agent\\
 \hline
Max2134 & testprivacy.com/1234  & Mozilla/5.0 (Linux; Android 9; SM-A505FN Build/PPR1.180610.011; wv) AppleWebKit/537.36 (KHTML, like Gecko) Version/4.0  Chrome/99.0.4844.88 Mobile Safari/537.36 Instagram 229.0.0.17.118  Android (28/9; 420dpi; 1080x2131;           samsung; SM-A505FN; a50; exynos9610;en\_DE; 360889116) \\ 
 \hline
Max2134 & testprivacy.com/1234  & Mozilla/5.0 (Linux; Android 9; SM-A505FN Build/PPR1.180610.011; wv) AppleWebKit/537.36 (KHTML, like Gecko) Version/4.0  Chrome/99.0.4844.88 Mobile Safari/537.36 Instagram 229.0.0.17.118  Android (28/9; 420dpi; 1080x2131;           samsung; SM-A505FN; a50; exynos9610;en\_DE; 360889116) \\ 
\hline
 \hline
Max2134 & testprivacy.com/1234  & Mozilla/5.0 (Linux; Android 9; SM-A505FN Build/PPR1.180610.011; wv) AppleWebKit/537.36 (KHTML, like Gecko) Version/4.0  Chrome/99.0.4844.88 Mobile Safari/537.36 Instagram 229.0.0.17.118  Android (28/9; 420dpi; 1080x2131;           samsung; SM-A505FN; a50; exynos9610;en\_DE; 360889116) \\ 
 \hline
\end{tabular}
 
 Case study 4 — Javascript modifying android objects
 
 
 Case study 5 — Unencrypted traffic
 In our dataset we have applications which sends uses unencrypted traffic to send user information. Private user information considered IP address and users google ads ID. As we this application sends private information. MITM easily can see and fingerprint the user information. 

 

\section{Related Work}


\section{Conclusion}


%
% ---- Bibliography ----
%
% BibTeX users should specify bibliography style 'splncs04'.
% References will then be sorted and formatted in the correct style.
%
 \bibliographystyle{splncs04}
 \bibliography{RelatedWork}
%
\end{document}
